%MIT OpenCourseWare: https://ocw.mit.edu
%RES.18-011 Algebra I Student Notes, Fall 2021
%License: Creative Commons BY-NC-SA 
%For information about citing these materials or our Terms of Use, visit: https://ocw.mit.edu/terms.

\section{Hilbert's Third Problem}

\subsection{Polygons in the Plane}

\begin{definition}
    Given polygons $P$ and $Q$ on the plane, $P$ is \textbf{scissors-congruent} to $Q$ (denoted $P \sim Q$) if we can divide $P$, using finitely many straight cuts, into a set of polygons $R_1$ through $R_n$; and we can divide $Q$ into the same collection $R_1$, \ldots, $R_n$. 
\end{definition}

\begin{example}
    This triangle and quadrilateral are scissors-congruent:
    \begin{center}
        \begin{asy}
            unitsize(2cm);
            draw((0, 0)--(2, 0)--(1, 0.5)--cycle);
            draw((1, 0)--(1, 0.5));

            draw((3, 0)--(3, 0.5)--(4, 0)--(4, -0.5)--cycle);
            draw((3, 0)--(4, 0));
        \end{asy}
    \end{center}
\end{example}

\begin{qq}
    Given two polygons $P$ and $Q$, when is $P \sim Q$?
\end{qq}

There's one obvious obstruction: we need $P$ and $Q$ to have the same area, since the area is preserved by rearranging pieces. This is actually an if and only if condition:

\begin{theorem}
    If $P$ and $Q$ have the same area, then $P \sim Q$. 
\end{theorem}

\begin{proof}[Proof Outline]
    The idea is to rearrange both $P$ and $Q$ to a rectangle with dimensions $1 \times \operatorname{area}(P)$, where $A$ is the common area -- if $P$ and $Q$ are scissors-congruent to the same rectangle, then they're scissors-congruent to each other. 
    
    First, cut $P$ into triangles $T_1$, \ldots, $T_s$. Then each triangle $T$ is scissors-congruent to some rectangle: cut the triangle as shown, and move the red pieces to the blue triangles. 
    \begin{center}
        \begin{asy}
            unitsize(2cm);
            pair A = (0, 1);
            pair B = (-0.7, 0);
            pair C = (1.3, 0);
            draw(A--B--C--cycle);
            draw(A--(0, 0));
            draw((A + B)/2--(A + C)/2);
            filldraw((-0.7, 0)--(-0.7, 0.5)--(-0.35, 0.5)--cycle, heavycyan+opacity(0.1), heavycyan);
            filldraw((1.3, 0)--(1.3, 0.5)--(0.65, 0.5)--cycle, heavycyan+opacity(0.1), heavycyan);
            filldraw((0, 1)--(-0.35, 0.5)--(0, 0.5)--cycle, heavyred+opacity(0.1), heavyred);
            filldraw((0, 1)--(0.65, 0.5)--(0, 0.5)--cycle, heavyred+opacity(0.1), heavyred);
        \end{asy}
    \end{center}
    Then we can show that any rectangle $R$ is scissors-congruent to another rectangle with dimensions $1 \times \operatorname{area}(R)$. This part is a bit finicky and involves a bunch of cases, so we'll skip it. 
    
    Finally, we have a bunch of triangles, which are each scissors-congruent to a rectangle of height $1$. So we can concatenate all these rectangles, to get a bigger rectangle of height $1$. This means $P$ is scissors-congruent to a $1 \times \operatorname{area}(P)$ rectangle, and by performing the same argument for $Q$, then $P \sim Q$. 
\end{proof}

\subsection{The Question}

We can now consider what happens in $3$ dimensions. 

\begin{definition}
    If $P$ and $Q$ are \textbf{polytopes} (meaning they have finitely many vertices, edges, and faces, and each face is a polygon), then $P \sim Q$ if you can use finitely many straight cuts to decompose $P$ and $Q$ into the same polytope pieces. 
\end{definition}

Again, there's an obvious condition: if $P \sim Q$, then they must have the same volume. 

\begin{qq}[Hilbert's Third Problem]
    If two polytopes have the same volume, are they scissors-congruent?
\end{qq}

In 1900, David Hilbert made a list of around twenty problems, which he considered the most important problems in modern mathematics. These problems were very influential.

This question was on that list, and he was pretty sure that the answer was no. In fact, this was the first one of his questions to be answered (in 1901), by his student Max Dehn. He showed more precisely that a cube and a tetrahedron of the same volume are not scissors-congruent. 

\subsection{Some Algebra}

At the heart of this problem is a certain algebraic construction. 

\begin{definition}
    Given two abelian groups $G$ and $H$, their \textbf{tensor product} $G \otimes H$ is the abelian group generated by elements of the form $g \otimes h$ for $g \in G$ and $h \in H$, satisfying the relations \[(g + g') \otimes h = g \otimes h + g' \otimes h,\] and similarly \[g \otimes (h + h') = g \otimes h + g \otimes h'.\] 
\end{definition}

You can think of $G \otimes H$ as taking \[\bigoplus_{g, h} \mathbb{Z}(g \otimes h)\] (which gives an integer for every pair $(g, h)$), and then quotienting out by the subgroup generated by $((g + g') \otimes h) - (g\otimes h) - (g' \otimes h)$ and so on. This gives a group; and because we've quotiented out by the subgroup, now our generators satisfy the given relations. 

\begin{question}
Is the direct sum $\bigoplus \mathbb{Z}(g \otimes h)$ the free group?
\end{question}

\begin{ans}
Not exactly -- it's a free \emph{abelian} group, since everything commutes. But if you'd like, you can instead start with the free group, and throw the condition that everything commutes into the quotient as well. 
\end{ans}

We can think of elements of $G \otimes H$ as expressions combining the terms $g \otimes h$, which we can simplify using the given relations. 

\begin{proposition}
The definition has a few consequences:
\begin{itemize}
    \item $0 \otimes h = g \otimes 0 = 0$.
    \item If $a \in \mathbb{Z}$, then $(ag) \otimes h = a(g \otimes h) = g \otimes ah$ (by using linearity). 
    \item If we take lists of generators $G = \langle g_1, \ldots, g_r\rangle$ and $H = \langle h_1, \ldots, h_s\rangle$, then $G \otimes H = \langle g_i \otimes h_j \rangle$.
\end{itemize}
\end{proposition}

\begin{example}
    We have $\mathbb{Z} \otimes G \cong G$. 
\end{example}

\begin{proof}
    We must have $(a \otimes g) \mapsto ag = 1 \otimes ag$. (So the isomorphism sends $(a \otimes g)$ to $ag$, and its inverse sends $g$ to $1 \otimes g$.)
\end{proof}

\begin{example}
    We have $(\mathbb{Z}^2) \otimes G \cong G \times G$. 
\end{example}

\begin{proof}
    We must have $(a, b) \otimes g \mapsto (ag, bg)$. 
\end{proof}

% \begin{example}
%     $\mathbb{Z} \otimes G \cong G$, since $a \otimes g \mapsto ag$. 

%     $\mathbb{Z}^{\oplus 2} \otimes G \cong G \times G$, since $(a, b) \otimes g \mapsto (ag, bg)$. 
% \end{example}

Note that in general, not everything in $G \otimes H$ is of the form $g \otimes h$: you can have expressions $g_1 \otimes h_1 + g_2 \otimes h_2$ which you can't simplify any further using these relations. So this isn't the same as looking at ordered pairs. 

\begin{example}
    We have $C_2 \otimes C_3 = 0$. 
\end{example}

\begin{proof}
    Take $x \otimes y$ for any $x \in C_2$ and $y \in C_3$. We have $3x = x$, so then \[x \otimes y = 3x \otimes y = x \otimes 3y = x \otimes 0 = 0,\] since $3y = 0$ in $C_3$.  
\end{proof}

So you can start with two nontrivial groups and take their tensor product, and get the $0$ group. If we just took the product, that would be a nontrivial group of size $6$. So the tensor product is somewhat subtle. 

\begin{question}
Does this happen with $2$ and $3$ replaced by any distinct positive integers?
\end{question}

\begin{ans}
Not necessarily distinct, but it does happen if the integers are relatively prime. 
\end{ans}

\subsection{Back to Polytopes}

If we have two polytopes of the same volume, we need another way to see they aren't scissors-congruent. 

Given a polytope $P$, each edge has a length $\ell$, and a \textbf{dihedral angle} $\theta$ -- where we take the perpendiculars to the edge on each face, and look at the angle between them. 

\begin{center}
    \begin{asy}
        import geometry;
        unitsize(2cm);
        filldraw((0, 0)--(0.6, 0.3)--(0.8, 1.8)--(0.2, 1.5)--cycle, gray+opacity(0.1), gray);
        filldraw((0, 0)--(0.6, 0.3)--(2.2, 0.1)--(1.6, -0.2)--cycle, gray+opacity(0.1), gray);
        draw((0, 0)--(0.6, 0.3), heavyred+linewidth(2));
        label("$\ell$", (0, 0), dir(225), heavyred);
        draw((0.5, 1.65)--(0.3, 0.15)--(1.9, -0.05), heavycyan);
        markangle("$\theta$", radius=-8, (0.5, 1.65), (0.3, 0.15), (1.9, -0.05), heavycyan);
    \end{asy}
\end{center}

Then $\ell$ and $\theta$ are real numbers, so we can take the tensor product \[\ell \otimes \theta \in \mathbb{R} \otimes \mathbb{R}/2\pi\mathbb{Z}.\] (Here $\mathbb{R}$ and $\mathbb{R}/2\pi\mathbb{Z}$ are both uncountably generated infinite groups.) Then every edge gives an element in this tensor product, so we can sum over the edges:

\begin{definition}
    The \textbf{Dehn invariant} of a polytope $P$ is \[d(P) = \sum_i \ell_i \otimes \theta_i \in \mathbb{R} \otimes \mathbb{R}/2\pi\mathbb{Z},\] where the sum is taken over all edges of $P$.  
\end{definition}

We'll see that the Dehn invariant is preserved by scissors-congruence -- then we can use it to tell whether two polytopes are scissors-congruent, similarly to how we can use volume. And it turns out that we can actually calculate it for some examples -- a cube and regular tetrahedron -- and see that they give different results. 

\begin{theorem}
    The Dehn invariant is preserved by scissors-congruence: if $P \sim Q$, then $d(P) = d(Q)$.
\end{theorem}

\begin{proof}
    Consider what happens to the polytope when we cut it. First, we'll look at what happens to the original edges.
    
    One possibility is that the edge is cut into two pieces, while the dihedral angle on each side remains the same:
    
    \begin{center}
        \begin{asy}
            import geometry;
            unitsize(3cm);
            draw((0.2, 1.5)--(0, 0), gray);
            draw((1, 0.5)--(1.2, 2), gray);
            fill((0, 0)--(1, 0.5)--(1.2, 2)--(0.2, 1.5)--cycle, gray+opacity(0.1));
            draw((1.6, -0.2)--(0, 0), gray);
            draw((1, 0.5)--(2.6, 0.3), gray);
            fill((1.6, -0.2)--(0, 0)--(1, 0.5)--(2.6, 0.3)--cycle, gray+opacity(0.1));
            draw((0, 0)--(1, 0.5), heavyred+linewidth(2));
            draw((0.5, 1.65)--(0.3, 0.15)--(1.9, -0.05), heavycyan);
            markangle("$\theta$", radius=-8, (0.5, 1.65), (0.3, 0.15), (1.9, -0.05), heavycyan);
            draw((0.8, 1.8)--(0.7, 0.35)--(2.1, 0.05), deepgreen+linewidth(1.3));
            fill((0.8, 1.8)--(0.7, 0.35)--(2.1, 0.05)--(2.2, 1.45)--cycle, deepgreen+opacity(0.1));
            label("$\ell_1$", (0.6, 0.3), dir(270), heavyred);
            label("$\ell_2$", (0.95, 0.475), dir(270), heavyred);
        \end{asy}
    \end{center}
    
    Then we have $\ell = \ell_1 + \ell_2$, so \[\ell \otimes \theta = \ell_1 \otimes \theta + \ell_2 \otimes \theta.\] 
    
    Another possibility is that we cut by a plane that contains this edge. Then the entire edge remains intact, but the angles change: now we have two polytopes each with an edge of length $\ell$, and angles $\theta_1$ and $\theta_2$ respectively. 
    
    \begin{center}
        \begin{asy}
            import geometry;
            unitsize(3cm);
            draw((0.2, 1.5)--(0, 0), gray);
            draw((1, 0.5)--(1.2, 2), gray);
            fill((0, 0)--(1, 0.5)--(1.2, 2)--(0.2, 1.5)--cycle, gray+opacity(0.1));
            draw((1.6, -0.2)--(0, 0), gray);
            draw((1, 0.5)--(2.6, 0.3), gray);
            fill((1.6, -0.2)--(0, 0)--(1, 0.5)--(2.6, 0.3)--cycle, gray+opacity(0.1));
            //draw((0.5, 1.65)--(0.3, 0.15)--(1.2, 1.05), heavycyan);
            markangle("$\theta_1$", radius=-13, (0.2, 1.5), (0, 0), (0.9, 0.9), heavycyan);
            draw((0, 0)--(1, 0.5)--(1.9, 1.4)--(0.9, 0.9)--cycle, deepgreen+linewidth(1.3));
            fill((0, 0)--(1, 0.5)--(1.9, 1.4)--(0.9, 0.9)--cycle, deepgreen+opacity(0.1));
            draw((0, 0)--(1, 0.5), heavyred+linewidth(2));
            draw((1.4, 1.15)--(0.5, 0.25)--(2.1, 0.05), heavycyan+dotted);
            markangle("$\theta_2$", radius=-12, (1.4, 1.15), (0.5, 0.25), (2.1, 0.05), heavycyan);
        \end{asy}
    \end{center}
    
    In this case, we have $\theta = \theta_1 + \theta_2$, so \[\ell \otimes \theta = \ell \otimes \theta_1 + \ell \otimes \theta_2.\] 
    
    In both cases, we used the linear property of the tensor product. 
    
    Meanwhile, we can also create new edges: when we make a cut along a face, this produces an edge on each of the two pieces. 
    
    \begin{center}
        \begin{asy}
            import geometry;
            unitsize(3cm);
            filldraw((0, 0)--(1.5, 0)--(1.7, 1)--(0.1, 0.9)--cycle, gray+opacity(0.1), gray);
            draw((1.3, 1.25)--(0.9, 0.95)--(0.8, 0)--(1.2, 0.3), deepgreen+linewidth(1.3));
            fill((1.3, 1.25)--(0.9, 0.95)--(0.8, 0)--(1.2, 0.3)--cycle, deepgreen+opacity(0.1));
            label("$\ell$", (0.85, 0.475), dir(180), heavyred);
            markangle("$\theta_1$", radius=-6, (0.1, 0.9), (0.9, 0.95), (1.3, 1.25), heavycyan);
            markangle("$\theta_2$", radius=-15, (1.3, 1.25), (0.9, 0.95), (1.7, 1), heavycyan);
        \end{asy}
    \end{center}
    
    Then we had a contribution of $0$ previously, and a contribution of $\ell \otimes \theta_1 + \ell \otimes \theta_2$ after the cut. But we have $\theta_1 + \theta_2 = \pi$, so \[\ell \otimes \theta_1 + \ell \otimes \theta_2 = \ell \otimes \pi = \frac{\ell}{2} \otimes 2\pi = 0\] (since the second group is $\mathbb{R}/2\pi\mathbb{Z}$).  
    
    The final case is if we have a new edge in the \emph{interior} of the polytope. Then we have a bunch of angles which add up to $2\pi$:
    
    \begin{center}
        \begin{asy}
            import geometry;
            unitsize(3cm);
            draw((0, 1)--(0.7, 0.8)--(0.7, -0.2)--(0, 0), gray);
            fill((0, 1)--(0.7, 0.8)--(0.7, -0.2)--(0, 0)--cycle, gray+opacity(0.1));
            draw((0, 1)--(-0.8, 0.7)--(-0.8, -0.3)--(0, 0), gray);
            fill((0, 0)--(0, 1)--(-0.8, 0.7)--(-0.8, -0.3)--cycle, gray+opacity(0.1));
            draw((0, 1)--(0.4, 1.4), gray);
            draw((0.4, 1.4)--(0.4, 0.4)--(0, 0), dotted+gray);
            fill((0, 0)--(0, 1)--(0.4, 1.4)--(0.4, 0.4)--cycle, gray+opacity(0.07));
            draw((0, 1)--(-0.6, 1.3), gray);
            draw((-0.6, 1.3)--(-0.6, 0.3)--(0, 0), dotted+gray);
            fill((0, 0)--(0, 1)--(-0.6, 1.3)--(-0.6, 0.3)--cycle, gray+opacity(0.07));
            draw((0, 0)--(0, 1), deepgreen+linewidth(1.3));
            markangle(radius=-13, (-0.8, 0.7), (0, 1), (-0.6, 1.3), heavycyan);
            markangle(radius=-6, (-0.6, 1.3), (0, 1), (0.4, 1.4), heavycyan);
            markangle(radius=-13, (0.4, 1.4), (0, 1), (0.7, 0.8), heavycyan);
            markangle(radius=-6, (0.7, 0.8), (0, 1), (-0.8, 0.7), heavycyan);
        \end{asy}
    \end{center}
    
    So the same thing happens: when we add up its contributions, we have \[\ell \otimes \theta_1 + \cdots + \ell \otimes \theta_k = \ell \otimes 2\pi = 0.\] So in all cases, the Dehn invariant is preserved by cutting the polytope. 
\end{proof}

\begin{question}
Does the argument used to show $\ell \otimes \pi = 0$ imply that \emph{any} $\ell \otimes \theta$ is $0$?
\end{question}

\begin{ans}
No -- we can only scale by integers. So this shows that if $\theta$ is a rational multiple of $\pi$, then $\ell \otimes \theta = 0$. But we'll see later that if $\theta$ isn't a rational multiple of $\pi$, we get something nonzero. But this is something to watch out for -- it happens a scary number of times that someone defines a complicated invariant, but it turns out the invariant is always $0$. 
\end{ans}

\begin{note}
We haven't thoroughly gone through all possibilities, but this gives a flavor of why the theorem is true -- in every case, if we compare the old and new contributations, then the relations we quotiented out by imply that the invariant doesn't change. 
\end{note}

Now we'll show that the invariant actually does something interesting. 

\begin{theorem}
    If $C$ is a cube and $T$ is a regular tetrahedron, then $d(C) = d(T)$. 
\end{theorem}

\begin{proof}
    A cube has $12$ edges, each of which has dihedral angle $\frac{\pi}{2}$. So we get \[d(C) = 12\ell \otimes \frac{\pi}{2} = \ell \otimes 6\pi = 0.\] So the Dehn invariant of a cube is always $0$. 

    For a regular tetrahedron, we have $6$ edges which are symmetric, so we get \[d(T) = 6\ell' \otimes \alpha\] for some angle $\alpha$. 
    
    \begin{center}
        \begin{asy}
            import geometry;
            unitsize(3cm);
            draw((0, 0)--(1.1, 0.1)--(0.4, -0.7)--cycle, gray);
            fill((0, 0)--(1.1, 0.1)--(0.4, -0.7)--cycle, gray+opacity(0.1));
            draw((0, 0)--(0.6, 0.6)--(0.4, -0.7)--cycle, gray);
            fill((0, 0)--(0.6, 0.6)--(0.4, -0.7)--cycle, gray+opacity(0.1));
            draw((0.6, 0.6)--(0.2, -0.35)--(1.1, 0.1), heavycyan);
            markangle("$\alpha$", radius=-8, (0.6, 0.6), (0.2, -0.35), (1.1, 0.1), heavycyan);
            draw((0.6, 0.6)--(0.6, 4/9*0.1 - 5/9*0.35), heavycyan);
        \end{asy}
    \end{center}

    To find $\alpha$, we can drop an altitude. This altitude ends at the center of the base:
    
    \begin{center}
        \begin{asy}
            unitsize(2cm);
            pair A = dir(90);
            pair B = dir(210);
            pair C = dir(330);
            draw(A--B--C--cycle);
            draw(A--(A + B + C)/3);
            draw(B--(C + A)/2);
            draw(C--(A + B)/2);
            draw((B + C)/2--(A + B + C)/3, heavycyan+linewidth(2.5));
        \end{asy}
    \end{center}
    
    So the short side of the right triangle is a third of the height of the base. 
    
    \begin{center}
        \begin{asy}
            import geometry;
            unitsize(3cm);
            pair A = (0, 0);
            pair B = (0.5, 0);
            pair C = (0, 1.5);
            draw(A--B--C--cycle);
            markangle("$\alpha$", radius=-8, C, A, B);
            label("$\frac{1}{3}h$", (0.25, 0), dir(270));
            label("$h$", (0, 0.75), dir(180));
        \end{asy}
    \end{center}
    Then, since the two triangles have the same height, we have \[\cos \alpha = \frac{1}{3}.\] 
    
    So we've found $\alpha$, and now we want to show that for this value of $\alpha$, the Dehn invariant is nonzero. 

    \begin{claim}
        $\alpha$ is not a rational multiple of $\pi$. 
    \end{claim}
    
    \begin{proof}[Proof Outline]
        One way to prove this is by trig identities -- suppose $\alpha = \frac{2a\pi}{b}$, and then use trigonometric identities to write \[\cos b\alpha = (\cos \alpha)^b2^b + \cdots.\] Then there's an enormous power of $3$ in the denominator, since $\cos \alpha = \frac{1}{3}$, so this can't be $1$.  
    \end{proof}

    \begin{claim}
        For any $\alpha \not\in \mathbb{Q}\pi$, and any $\ell \in \mathbb{R}$, $\ell \otimes \alpha$ is nonzero. 
    \end{claim}

    \begin{proof}
        Think of $\mathbb{R}$ as a vector space over $\mathbb{Q}$ (of uncountable dimension). Then $\pi$ and $\alpha$ are linearly independent, so we can fill them out into a basis: we can write \[\mathbb{R} = \mathbb{Q}\alpha \oplus \mathbb{Q}\pi \oplus W,\] where $W$ comes from the uncountably many remaining basis elements. Then we can define a linear map $f : \mathbb{R} \to \mathbb{Q}$ (a map of $\mathbb{Q}$-vector spaces) sending $\alpha$ to $1$, and every other basis element to $0$ -- we can do this because to define a linear map, it suffices to define it on each basis element. 
        
        Then we have a group homomorphism $g : \mathbb{R} \otimes \mathbb{R}/2\pi\mathbb{Z} \to \mathbb{R}$, where $z \otimes x \mapsto z f(\tilde{x})$, where $\tilde{x}$ is $x$ mod $2\pi$ -- since $2\pi \in \ker(f)$, this is well-defined. (We can also check that this is compatible with the relations of the tensor product.)
        
        Now we have $g(\ell \otimes \alpha) = \ell \neq 0$. And since $g(\ell \otimes \alpha)$ is nonzero, then $\ell \otimes \alpha$ must be nonzero as well.
    \end{proof}

    So $d(T) = \ell' \otimes \alpha$ is nonzero, which means $d(C) \neq d(T)$. 
\end{proof}

The Dehn invariant lies in some enormous group. In this proof, in order to show it was nonzero, we mapped it to a much simpler group -- the real numbers -- and showed instead that its \emph{image} is nonzero. 

\begin{note}
This was proven in 1901. In 1968 Sydler showed the converse -- that if $P$ and $Q$ have the same volume \emph{and} Dehn invariants, then they're scissors-congruent. This is known in dimension $4$ as well. But in higher dimensions, it's not actually known how to characterize when two polytopes are scissors-congruent. 
\end{note}
\newpage